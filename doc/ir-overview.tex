
\subsection{\vine Internal Representation (IR)}

\begin{table}
\begin{tabular}{lll}
  stmt &::=& {\tt Jmp} of exp $|$ {\tt CJmp} of exp * exp * exp \\
  & & $|$ {\tt Move} of lvalue * exp  $|$ {\tt Special} of string \\
  & & $|$ {\tt Label} of label $|$ {\tt Block} of decl list * stmt list\\
  & & $|$ {\tt ExpStmt} of exp $|$ {\tt Comment} of string  \\
  & & $|$ {\tt Function} of var * typ * decl list * bool * stmt option
  \\
  & & $|$ {\tt Call} of lvalue option * var * exp list \\
  & & $|$ {\tt Attr} of stmt * attribute\\

  exp &::=& {\tt BinOp} of  binop\_type * exp * exp
  $|$ {\tt UnOp} of unop\_type * exp \\
  & & $|${\tt Constant} of typ * value 
  $|$ {\tt Name} of label \\
& &   $|$ {\tt Cast} of cast\_type * reg\_type * exp\\
  & &$|$ {\tt Unknown} of string 
  $|$ {\tt Lval} of lvalue \\
  & & $|$ {\tt Let} of lvalue * exp * exp \\

  lvalue &::=& {\tt Temp}  of var * typ 
  $|$ {\tt Mem} of var * typ * exp \\

  binop\_type &::=& arith\_bops $|$ rel\_bops \\
  arith\_bops&::=& PLUS ($+$) $|$ MINUS ($-$) $|$ TIMES ($*$) $|$ DIVIDE
  ($/$)  $|$ SDIVIDE ($/\$$) $|$ MOD ($\%$) \\
  & & $|$ LSHIFT ($\ll$) $|$ RSHIFT ($\gg$) $|$ ARSHIFT ($@\gg$) 
   $|$ AND ($\&$) $|$ OR ($|$) $|$ XOR (\^{}) \\
  rel\_bops&::=&  EQ $(==)$ $|$ NEQ $(<>)$ $|$ $|$ LT ($<$) $|$
  SLE ($<=\$$) $|$ SLT ($<\$$) \\
  unop\_type&::=& NEG ($-$) $|$ NOT ($!$)\\

  typ&::=& reg\_t $|$ {\tt Array} of typ * typ $|$ {\tt TAttr} of
  attributes * typ\\

  reg\_t&::=& REG\_64 $|$ REG\_32 $|$ REG\_16 $|$ REG\_8 $|$ REG\_1\\
  cast\_type&::=&CAST\_UNSIGNED $|$ CAST\_SIGNED $|$ CAST\_HIGH $|$
  CAST\_LOW \\
  & & $|$ CAST\_FLOAT $|$ CAST\_INTEGER $|$ CAST\_RFLOAT $|$
  CAST\_RINTEGER \\

  decl&::=& var * typ\\
  var&::=& string \\
  label&::=& string 
 
\end{tabular}
\caption{Our IR Constructors}
\label{tab:irsyntax}
\end{table}


Our IR is shown in Table~\ref{tab:irsyntax}. Our IR consists of
statements (stmt) and expressions (exp).  Expressions are pure, i.e.,
side-effect free. One thing to note is that both jumps ({\tt Jmp}) and
conditional jumps ({\tt CJmp}) have targets that are expressions, not
necessarily labels.  That is because the jump target may be indirect,
i.e., calculated via an expression.~\footnote{One may wonder why we
have {\tt Label} and {\tt Name}. Both name a location in the
program. The reason we need {\tt Name} as an expression is because
jump targets should be expressions, i.e., they may be calculated, or
they may be a known location given by {\tt Name}. However, we want
program locations to be ``higher level'' than expressions, so we also
have {\tt Label}. Thus, we have {\tt Name} of {\tt Label}, i.e., an
expression containing a statement. This may seem weird, but since {\tt
Label} has no side effects, it still is at least consistent in some
respect.}.

Statements in our language are:
\begin{itemize}\squish

\item {\tt Jmp}({\it exp}) - {\tt Jmp} is an unconditional jump. {\it
  exp} is the jump target. If {\it exp} is of type {\tt Name}, then
  the jump is to a known location, i.e., a direct jump.  If {\it exp}
  is not of type {\tt Name}, then the jump is indirect.

\item {\tt CJmp}({\it exp}, {\it true\_exp\_target}, {\it
  false\_exp\_target}) - {\tt CJmp} is a conditional jump.  {\it exp}
  is evaluated, and if true, control is transfered to {\it
  true\_exp\_target}, else control is tranfered to {\it
  false\_exp\_target}.  When constructing a {\tt CJmp}, {\it exp}
  should evaluate to type bool, though we do not explicitly
  check. {\it true\_exp\_target} and {\it false\_exp\_target} are
  similar to the expression in {\tt Jmp}: if they are of type {\it
  Name}, then the jump target is known, else the jump target is
  indirect. Note that in binaries, usually {\it false\_exp\_target}
  will be a fall-through address, thus a known location, thus a {\tt
  Name}.

\item {\tt Move}({\it lhs\_exp}, {\it rhs\_exp}) - {\tt Move} is our
  assignment statement {\it lhs\_exp} := {\it rhs\_exp}. {\tt Move} is
  used for both load and store, i.e, if {\it lhs\_exp} is a {\tt Mem},
  then the {\tt Move} is a store to memory, and if {\it lhs\_exp} is
  a {\tt Temp}, then this is a load.

\item {\tt Call}({\it lvalue option}, {\it var}, {\it exp list}) - calls
  the function named by {\it id} with arguments {\it exp list}. The
  call can optionally return a value. Functions with no return have
  {\tt None} as the {\it lvalue option}, otherwise will have {\tt
  Some(x)} where the statement can be read as {\tt x = id(arg1, arg2,
  ... ,argn)}.

\item {\tt Function}({\it var},  {\it typ}, {\it decl list}, {\it
  bool}, {\it stmt option}) - if {\it bool} is false, this is an
  external function.  If {\it stmt option} = {\tt None}, then this is
  a declaration. If {\it stmt option} = {\tt Some}, then this is a
  function definition.  The function name is {\it var}, with return
  type {\it typ}, and formal arguments named by {\it decl list}. We
  require formals to have names and types, unlike C (which just
  requires the types).

\item {\tt Special}({\it string}) - {\tt Special} are for
  instructions that change the processor state, such as halt,
  interrupts, etc.  The {\it string} is the x86 instruction name. 

\item {\tt Label}({\it exp}) - A label is an abstract location in the
  program, e.g., the beginning of a basic block. Labels serve as
  targets for jumps (both {\tt Jmp} and {\tt CJmp}).  Direct jumps
  will have {\it exp} be a {\sc Name}, while indirect jumps will have
  some arithmetic expression. Labels have no effect on execution.

\item {\tt ExpStmt}({\it exp}) - An {\tt ExpStmt} is a statement which
  executes an expression, then throws away the result. {\tt ExpStmt}
  is useful for anaysis: you will not see a direct translation of an
  x86 instruction to {\tt ExpStmt}.

\item {\tt Comment}({\it string}) - A {\tt Comment} is a user-written
  comment in the code. Comments may be inserted by analysis, during
  translation, etc, and are used to make the code more
  readable. Comments have no effect on execution.
\end{itemize}
