\section{The IR}\label{sec:IR}

The IR is the front-end for all our research. The IR library is
responsible for translating the binary into a usable and unambigious
IR.  Our IR, called tigger, is an adaption of the IR used in the Tiger
compiler book~\cite{appelBOOK:1998}.

\subsection{The IR}

\begin{table}
\begin{tabular}{lll}
  stmt &::=& {\tt Jmp} of exp $|$ {\tt CJmp} of exp * exp * exp \\
  & & $|$ {\tt Move} of lvalue * exp  $|$ {\tt Special} of string \\
  & & $|$ {\tt Label} of label $|$ {\tt Block} of decl list * stmt list\\
  & & $|$ {\tt ExpStmt} of exp $|$ {\tt Comment} of string  \\
  & & $|$ {\tt Function} of label * type *  decl list * stmt \\
  & & $|$ {\tt Return} of exp option $|$ {\tt Attr} of stmt * attribute\\
  & & $|$ {\tt Call} of lvalue option  * label * exp list\\

  exp &::=& {\tt BinOp} of  binop\_type * exp * exp
  $|$ {\tt UnOp} of unop\_type * exp \\
  & & $|${\tt Constant} of typ * value 
  $|$ {\tt Name} of label \\
  & & $|$ {\tt Phi} of var * exp list 
  $|$ {\tt Cast} of cast\_type * typ * exp\\
  & & $|$ {\tt Unknown} of string 
  $|$ {\tt Lval} of lvalue \\
  & & $|$ {\tt Let} of lvalue * exp * exp \\

  lvalue &::=& {\tt Temp}  of var * typ 
  $|$ {\tt Mem} of var * typ * exp \\

  binop\_type &::=& arith\_bops $|$ rel\_bops \\
  arith\_bops&::=& PLUS ($+$) $|$ MINUS ($-$) $|$ TIMES ($*$) $|$ DIVIDE
  ($/$)  $|$ SDIVIDE ($/\$$) $|$ MOD ($\%$) \\
  & & $|$ LSHIFT ($\ll$) $|$ RSHIFT ($\gg$) $|$ ARSHIFT ($@\gg$) 
   $|$ AND ($\&$) $|$ OR ($|$) $|$ XOR (\^{}) \\
  rel\_bops&::=&  EQ $(==)$ $|$ NEQ $(<>)$ $|$ $|$ LT ($<$) $|$
  SLE ($<=\$$) $|$ SLT ($<\$$) \\
  unop\_type&::=& NEG ($-$) $|$ NOT ($!$)\\

  typ&:==& REG\_64 $|$ REG\_32 $|$ REG\_16 $|$ REG\_8 $|$ REG\_1 $|$ {\tt Array} of typ * typ $|$ String $|$ Void\\

  cast\_type&::=&CAST\_UNSIGNED $|$ CAST\_SIGNED $|$ CAST\_HIGH $|$
  CAST\_LOW \\
  & & $|$ CAST\_FLOAT $|$ CAST\_INTEGER $|$ CAST\_RFLOAT $|$
  CAST\_RINTEGER \\

  decl&::=& var * typ\\
  var&::=& string \\
  label&::=& string 
 
\end{tabular}
\caption{Our IR Constructors}
\label{fig:irtypes}
\end{table}


Our IR is shown in Table~\ref{fig:irtypes}. Our IR consists of
statements (Stmt) and expressions (Exp).  Expressions are pure, i.e.,
side-effect free. One thing to note is that both jumps ({\tt Jmp}) and
conditional jumps ({\tt CJmp}) have targets that are expressions, not
necessarily labels.  That is because the jump target may be indirect,
i.e., calculated via an expression.~\footnote{One may wonder why we
have {\tt Label} and {\tt Name}. Both name a location in the
program. The reason we need {\tt Name} as an expression is because
jump targets should be expressions, i.e., they may be calculated, or
they may be a known location given by {\tt Name}. However, we want
program locations to be ``higher level'' than expressions, so we also
have {\tt Label}. Thus, we have {\tt Name} of {\tt Label}, i.e., an
expression containing a statement. This may seem weird, but since {\tt
Label} has no side effects, it still is at least consistent in some
respect.}.

%% Statements in our language are:
%% \begin{itemize}\squish

%% \item {\tt Jmp}({\it exp}) - {\tt Jmp} is an unconditional jump. {\it
%%   exp} is the jump target. If {\it exp} is of type {\tt Name}, then
%%   the jump is to a known location, i.e., a direct jump.  If {\it exp}
%%   is not of type {\tt Name}, then the jump is indirect.

%% \item {\tt CJmp}({\it exp}, {\it true\_exp\_target}, {\it
%%   false\_exp\_target}) - {\tt CJmp} is a conditional jump.  {\it exp}
%%   is evaluated, and if true, control is transfered to {\it
%%   true\_exp\_target}, else control is tranfered to {\it
%%   false\_exp\_target}.  When constructing a {\tt CJmp}, {\it exp}
%%   should evaluate to type bool, though we do not explicitly
%%   check. {\it true\_exp\_target} and {\it false\_exp\_target} are
%%   similar to the expression in {\tt Jmp}: if they are of type {\it
%%   Name}, then the jump target is known, else the jump target is
%%   indirect. Note that in binaries, usually {\it false\_exp\_target}
%%   will be a fall-through address, thus a known location, thus a {\tt
%%   Name}.

%% \item {\tt Move}({\it lhs\_exp}, {\it rhs\_exp}) - {\tt Move} is our
%%   assignment statement {\it lhs\_exp} := {\it rhs\_exp}. {\tt Move} is
%%   used for both load and store, i.e, if {\it lhs\_exp} is a {\tt Mem},
%%   then the {\tt Move} is a store to memory, and if {\it lhs\_exp} is
%%   a {\tt Temp}, then this is a load.

%% \item {\tt Special}({\it string}) - {\tt Special} are for
%%   instructions that change the processor state, such as halt,
%%   interrupts, etc.  The {\it string} is the x86 instruction name. 

%% \item {\tt Label}({\it exp}) - A label is an abstract location in the
%%   program, e.g., the beginning of a basic block. Labels serve as
%%   targets for jumps (both {\tt Jmp} and {\tt CJmp}).  Direct jumps
%%   will have {\it exp} be a {\sc Name}, while indirect jumps will have
%%   some arithmetic expression. Labels have no effect on execution.

%% \item {\tt ExpStmt}({\it exp}) - An {\tt ExpStmt} is a statement which
%%   executes an expression, then throws away the result. {\tt ExpStmt}
%%   is useful for anaysis: you will not see a direct translation of an
%%   x86 instruction to {\tt ExpStmt}.

%% \item {\tt Comment}({\it string}) - A {\tt Comment} is a user-written
%%   comment in the code. Comments may be inserted by analysis, during
%%   translation, etc, and are used to make the code more
%%   readable. Comments have no effect on execution.
%% \end{itemize}

%% Expressions in our language are:
%% % FIXME: why is this duplicated
%% \begin{itemize}\squish

%% \item {\tt BinOp}({\it optype}, {\it lhs}, {\it rhs}) -  A binary
%%   operation.  

%% \item {\tt UnOp}({\it optype}, {\it exp}) - A unary operation.

%% \item {\tt Constant}({\it regtype}, {\it value}) - A constant.

%% \item {\tt Temp}({\it regtype}, {\it string}) - An abstract
%%   register. There are an infinite number of Temps.

%% \item {\tt Name}({\it label}) - An abstract name. 

%% \item {\tt Mem}({\it exp}) - A memory reference to address {\it exp}.

%% \item {\tt Phi}( {\it var} list) - {\tt Phi} expressions are used for
%%   SSA. 

%% \item {\tt Cast}({\it casttype}, {\it exp}) - A cast.

%% \item {\tt Unknown}({\it string}) - An instruction we could not
%%   translate. The {\it string} is the assembly.  Note that although {\tt
%%   Unknown}'s are part of the IR, we consider them an artifact of
%%   incomplete translation.  Thus, for the most part, we ignore them in
%%   our formalism, though certainly any implementation may have to make
%%   a choice of what to do when an unknown is encountered.

%% \item {\tt Extension} - This is a place-holder for extensions to the
%%   IR, i.e., for adding a new expression type. 
%% \end{itemize}

%% {\tt binop\_type} and {\tt unop\_type} are mostly straightforward: they are
%% the binary operation and unary operation. Note that for {\tt
%%   unop\_type} the difference between NOT and BNOT is the latter is a
%% bitwise-not  while the former is a logical NOT.  {\tt register\_type}
%% is the type of register, and consists of both the kind {\tt
%%   reg\_kind} and width {\tt reg\_width}.  {\tt cast\_type} is the type
%% of cast, and is a carry-over from valgrind cast types where more
%% information can be found.

\subsection{IR Static Semantics}
% IR static semantics

%% Metavariables used: $i$ for immediate value, $v$ for variable name,
%% $e$ for expression, $\tau$ for type.

\subsubsection{Types}
Our basic types are: {\tt uint64\_t}, {\tt uint32\_t}, {\tt
uint16\_t}, {\tt uint8\_t}, {\tt int64\_t}, {\tt int32\_t}, {\tt
int16\_t}, {\tt int8\_t}, {\tt Bool}, and {\tt address\_t}. 


We maintain a typing context $\Gamma$, which is simply a sequence of
variables (i.e., Temp's) and their types (denoted $v:\tau$ for $v$ is
of type $\tau$).  In our typing rules below ``()'' denotes the unit
type, and ``\_'' denotes we don't care about the value. The rules can
be read as an algorithm: we can conclude the bottom if we check the
top, e.g., the type of Move is unit if we first check $e_1$ and see if
it is a valid l-value, then make sure $e_2$ is a valid type.

{\it Note the following rules are slightly broken (and will be
fixed). However, they should for the most part give you a good idea
what we expect when forming expressions and statements.}

{\bf Expression Types}: \\
\[
\setlength\extrarowheight{15pt}
\begin{array}{cc}
% In array, add extra space in-between rows.
\infer[\textsc{t-temp}]{\Gamma \vdash \text{Temp}(\tau,x):\tau}
{ \text{Temp}(\tau, x) \in \Gamma} &
\infer[\textsc{t-const}]{\Gamma \vdash \text{Constant}(\tau,x):\tau}{}\\
\infer[\textsc{t-unop}]{\Gamma \vdash \text{UnOp}(\circ,t_1) : \tau}
{\Gamma \vdash t_1 : \tau & \circ \in \text{unop\_type}} &
\infer[\textsc{t-name}]{\Gamma \vdash \text{Name(\_)} : address\_t}{}\\
\infer[\textsc{t-phi}]{\Gamma \vdash \text{Phi}(\_, e) :
\tau} {\Gamma \vdash e : \text{Temp list} &  \forall x \in e | \Gamma
\vdash x:\tau } &
\infer[\textsc{t-unknown}]{\Gamma \vdash \text{Unknown}(\_) :\top}{} \\
\infer[\textsc{t-mem}]{\Gamma \vdash \text{Mem}(\tau, e) :
address\_t}{\Gamma \vdash e: address\_t}
& 
\infer[\textsc{t-address}]{uint32\_t : address\_t}{}
\\
\end{array}
\]
\[
\setlength\extrarowheight{15pt}
\begin{array}{c}
\infer[\textsc{t-binop-arith}]{\Gamma \vdash \text{BinOp}(\circ,t_1,
t_2):\tau} 
{\Gamma \vdash t_1 :\tau &  \Gamma \vdash t_2:\tau  & 
\circ \in \text{arith\_bops}}\\
\infer[\textsc{t-binop-bool}]{\Gamma \vdash \text{BinOp}(\circ,t_1,
t_2) : \bool}{\Gamma \vdash t_1 :\tau &  \Gamma \vdash t_2:\tau  &{\circ \in \text{rel\_bops}} }\\
\end{array}
\]
\\
{\bf Statement Types:}
\[
\setlength\extrarowheight{15pt}
\begin{array}{cc}
\infer[\textsc{t-jmp}]{\Gamma \vdash \text{Jmp}(e) : ()}
{ \Gamma \vdash e : \text{Name} \vee e : address\_t} &
\infer[\textsc{t-cjmp}]{\Gamma \vdash \text{CJmp}(e_1, e_2, e_3) : ()}
{\Gamma \vdash e_1:\text{Bool} & \Gamma \vdash e_1, e_2  : address\_t} \\
\infer[\textsc{t-move}]{\Gamma \vdash \text{Move}(e_1, e_2) :() }
{\Gamma \vdash e_1 : LVal & \Gamma \vdash e_2 :
\tau} &
\infer[\textsc{t-special}]{\Gamma \vdash \text{Special}(e) :
()}{}\\
\infer[\textsc{t-label}]{\Gamma \vdash \text{Label}(e) : ()}
{\Gamma \vdash e:\text{string}} &
\infer[\textsc{t-expstmt}]{\Gamma \vdash \text{ExpStmt}(e) : () }
{\Gamma \vdash e:\tau} \\
\infer[\textsc{t-comment}]{\Gamma \vdash \text{Comment}(e) : () }
{ \Gamma \vdash e : \text{string}}\\
\end{array}
\]

%% \subsubsection{Types of Expressions}

%% \[
%% \begin{array}{c}
%% \infer[\text{T-Var}]{\Gamma \vdash x:\tau}{\text{Temp}(x,\tau) \in \Gamma}\\
%% \infer[\text{T-Unsigned}]{\text{uint8\_t} <: \text{uin16\_t} <: uint32\_t <:
%% uint64\_t}{} \\
%% \infer[\text{T-Signed}]{\text{int8\_t} <: ... }{}\\
%% \infer[\text{T-BinOp}]{BinOp(op, x_1, x_2) : \tau_2}{\Gamma,x_1:\tau_1 &
%% \Gamma,x_2:\tau_2 & \tau_1 <: \tau_2 }\\
%% \end{array}
%% \]


%% \begin{mathpar}
%% \infer[Const]{ } { \mathtt{Constant}(i : \tau) : \tau}
%% \and
%% \infer[Temp]{ } { \mathtt{Temp}(v : \tau) : \tau}
%% \and
%% \infer[Name]{ } {\mathtt{Name}(v) : (UNSIGNED, ???)}
%% \and
%% % Um, how do we say \tau is any integer type?
%% \infer[Binop-Int]{\circ \in \{ MOD, LSHIFT, RSHIFT, ARSHIFT, LROTATE, RROTATE, LOGICAND, LOGICOR, BITAND, BITOR, XOR\} \\ e_1 : \tau \\ e_2 : \tau \\ \tau \in Integer \cup Unsgined} { e_1 \circ e_2 : \tau}
%% \and
%% \infer[Binops-Arr]{\circ \in \{ PLUS, MINUS, TIMES, DIVIDE,\} \\ e_1 : \tau \\ e_2 : \tau} { e_1 \circ e_2 : \tau}
%% \and
%% \infer[Binops-Rel]{\circ \in \{ EQ, NEQ, GT, LT, GE, LE \} \\ e_1 : \tau \\ e_2 : \tau} {e_1 \circ e_2 : \mathtt{Bool}}
%% \and
%% \infer[Unop]{\diamond \in \{ NEG, NOT, BNOT \} \\ e : \tau} {\diamond e : \tau}
%% \and
%% \infer[Phi]{ e_0 : \tau \\ ... \\ e_n : \tau} {\Phi_v(e_0, ..., e_n) : \tau}
%% \and
%% \infer[Mem]{e : \tau \\ \tau = (UNSIGNED, ???)} {\mathtt{Mem}(e, \tau') : \tau'}
%% \and
%% \infer[Cast-Wide]{c \in \{ CAST\_SIGNED, CAST\_UNSIGNED \} \\ e : (k???,w) \\ w < w'} {\mathtt{Cast}(c,w',e) : (k,w')}
%% \and
%% \infer[Cast-Narrow]{c \in \{ CAST\_LOW, CAST\_HIGH\} \\ e : (k,w) \\ w > w' }{\mathtt{Cast}(c, w', e) : (k,w')}
%% \and
%% \infer[Cast-Int]{c \in \{ CAST\_INTEGER, CAST\_RINTEGER \} \\ e : (FLOAT,w) } {\mathtt{Cast}(c,w,e) : (INTEGER,w)}
%% \and
%% \infer[Cast-Float]{c \in \{ CAST\_FLOAT, CAST\_RFLOAT \} \\ e : (k,w) } {\mathtt{Cast}(c,w,e) : (FLOAT,w)}
%% \end{mathpar}


%% \subsubsection{Types of statements}
%% \begin{mathpar}
%% \infer[Label]{ }{\mathtt{Label} s : ()}
%% \and
%% \infer[Comment]{ }{\mathtt{Comment} s : ()}
%% \and
%% \infer[Special]{ }{\mathtt{Special} s : ()}
%% \and
%% \infer[Jmp]{e:Address} {\mathtt{Jmp}(e) : ()}
%% \and
%% \infer[CJmp]{e_1:Bool\\ e_2:Address \\ e_3:Address}{\mathtt{CJmp}(e_1, e_2, e_3) : ()}
%% \and
%% \infer[Move]{e_1:\tau \\ e_2 : \tau \\ e_1 \in \{ \mathtt{Mem}, \mathtt{Temp} \} } {\mathtt{Move}(e_1,e_2) : ()}
%% \and
%% \infer[ExpStmt]{e : \tau}{\mathtt{ExpStmt} e : ()}
%% \end{mathpar}


\subsection{IR Operational Semantics}

To be written

\subsection{IR expression comparisons}

For the purpose of determining whether two expressions are equivalent,
we compare their canonicalized forms for structural equality. This is
used for eg. alias analysis.

The cannonical order for subexpressions is:
\begin{enumerate}
\item Constants (ordered in descending order by value) %first
\item Name (ordered alphabetically)
\item Temp (ordered alphabetically)
\item Mem (ordered by their addresses)
\item FIXME: define the others
\item Unknown (ordered alphabetically) %last
\end{enumerate}
Note: \texttt{Let} is excluded from the ordering, because it will not
appear in subexpressions.

An expression is in canonical form when:
\begin{itemize}
\item
The only \texttt{Let}s are \texttt{Let}s which may alias part of its
subexpression, but only in cases when it cannot be determined that
there is definite aliasing.

\item
All \texttt{Let}s are on the outside of the expression, ordered in the
cannonical order by their addresses, from outside to inside.

\item
All associative binops, are associated to the right.

\item
  All commutative binops have their subexpressions in the cannonical order, from right to left:


\item
No more constant folding can be performed.

\item
It contains no product of sums.


\item
FIXME: Need more.
\end{itemize}


\subsection{Using the IR}

In this section I describe using visitors.

%% \subsection{Implemented IR Routines}

%% In this section we describe IR routines already implemented as part of
%% the IR library. 

%% \subsubsection{Control Flow Graph (CFG) and Call Graph (CG)}

%% \subsubsection{SSA Form}

%% \subsubsection{Dead Code Elimination}


\subsection{IR Creation}

Our IR is created from a list of assembly instructions.  The relevant
headers are {\tt stmt.h}, {\tt exp.h}, {\tt irtoir.h}, and {\tt
vexmem.h}. Normally, one would define the transformation rules for
each x86 instruction and it's corresonding IR.  We take a short-cut.

We pass each instruction in the instruction list to
Valgrind~\cite{net:valgrind2003}, an x86 emulator. Valgrind will
translate the assembly instruction into its own IR.  We then take the
Valgrind IR and translate it to our own IR (via the code in the {\tt
  vex} libasmir director). 

One may wonder why we translate to our own IR if we have Valgrind's
IR. This is a good question.  We have learned that IR's for dynamic
execution are often insufficient for static analysis.  The central
reason is a dynamic IR can get away with being imprecise by not
modelling certain characteristics of the CPU. For example, the
Valgrind IR never calculates EFLAGS (the status registers) directly,
since it can simply execute the code to figure out which status flags
have changed.  So, the high level answer is dynamic IR's seem to be
imprecise and potentially ambigious, while our IR is precise and
unambiguous.


%% We currently support ELF executables with symbol tables. We statically
%% compile all our test cases.  Support for PE and stripped binaries is
%% straight-forward: users will only need to add support for locating
%% function boundries (e.g., by looking for function prologues and
%% epilogues).

%% The steps for creating the IR are:
%% \begin{itemize}
%% \item Parse the binary into a series of sections. Locate executable
%%   sections.
%% \item Locate functions within each section.
%% \item Disassemble each function.
%% \item Convert each assembly instruction to the IR.
%% \end{itemize}
%% We describe the details in more depth below.

%% \subsubsection{Parsing the Binary}
%% We use GNU BFD to parse the binary.  We disassemble all sections with
%% SEC\_CODE set as a flag.  

%% \subsubsection{Locate functions}
%% If the symbol table is provided, we also locate it.



