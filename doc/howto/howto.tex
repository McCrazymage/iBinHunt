\documentclass[11pt,onecolumn]{article}
\usepackage{fancyvrb}
\usepackage{relsize}
\usepackage{fullpage}
\usepackage{amstext}
\usepackage[dvips]{graphicx}       %%% graphics for dvips
\usepackage[colorlinks=true, linkcolor=blue, 
  citecolor=blue, urlcolor=blue,
  ps2pdf,                %%% hyper-references for ps2pdf
  bookmarks=true,        %%% generate bookmarks ...
  bookmarksnumbered=true,%%% ... with numbers
]{hyperref}
% pdfcreator and pdfproducer are set automatically in pdfLaTeX
\hypersetup{ pdfcreator  = {LaTeX with hyperref package},
  pdfproducer = {dvips + ps2pdf} }
\let\url\nolinkurl % because dvips cannot break url across lines
%\usepackage{times}
% Make \_ be CMTT's underscore, not \textunderscore, since we use
% it in code and command names.
\def\_{\char"5F}

\newcommand{\comment}[1]{}
\newcommand{\mytt}{\small \tt}
\newcommand{\titled}{How to Install and Run an Example with VinE}
\title{\mbox{}\\[-.8in]\bf \titled}
\author{Original version by Prateek Saxena, with updates from Stephen McCamant}
\date{Sep 12th, 2008: SVN trunk r3466 and Ubuntu 8.04}
\begin{document}
\maketitle

This document is intended to be a quick start guide for setting up and
running VinE, the static analysis component of the BitBlaze Binary
Analysis Framework. It assumes that you have some familiarity with
Linux.  The instructions are based on the version of VinE in the SVN
trunk as of the date shown in the header, running on a vanilla Ubuntu
8.04 distribution of Linux.  The procedure is intermixed with
explanations about utilities to give an overview of how things
work. The goal in this exercise is to trace from a simple program with
symbolic keyboard input, and generate an STP file which models the
weakest precondition of the control-flow path the program took. In
other words, the conditions on the inputs that cause it to take a
execute a certain branch of code. To follow along with the
instructions, you'll need to start with a trace file like the one
generated in the separate TEMU tutorial.

\section {Installation}
\label{sec:install}

The following script, which is also found as
\verb'docs/install-vine.sh' in the VinE source, shows the steps for
building and installing Vine and the other software it depends on:

\VerbatimInput{../install-vine.sh}

\section {Generating the IR and the STP formula}

We start with a trace (generated, for instance, by TEMU) that records
the instructions executed on a program run, the data values they
operated on, and which data values were derived from a distinguished
set of (``tainted'') input values. We're going to do operations where
we consider that input to be a symbolic variable, but the first step
is to interpret the trace. The x86 instructions in the trace are a
pretty obscure representation of what is actually happening in the
program, so we'll translate them into a cleaner intermediate
representation (IR) language.

First, let us check if we have got a meaningful trace.  One way to do
so is to print the trace, and see that at least the expected
instructions are marked as tainted.  For this, you may use the
\texttt{trace\_reader} command utility in Vine. As shown below, in the
output you should be able to see the compare (or similar) instruction
that comapares the input to the immediate value 5. The presence of
tainted operands in any instruction are indicated by the record
containing ``T1''.

\begin{Verbatim}[fontsize=\relsize{0}, frame=lines, framesep=.5em]

% cd bitblaze/vine
% ./utils/trace_reader -trace /tmp/foo.trace | grep T1
...
...
804845a:       cmpl    $0x5,-0x4(%ebp)   I@0x00000000[0x00000005] \
       T0      M@0xbffffac4[0x00000005]        T1 {15 (1001, 0) (1001, 0) \
	 (1001, 0) (1001, 0) } 

\end{Verbatim}
%$

Of course, the real output of that command contains lots of
instructions, but we've picked out a key one: an instruction from the
main program (you can tell because the address is in the \verb'0x08000000'
range) in which a value from the stack (\verb'-0x4(%ebp)') is
compared (a \verb'cmpl' instruction) with a constant integer 5
(\verb'$0x5').%$
The later fields on the line represent the instruction operands and
their tainting.

We can then use a single program, the \texttt{appreplay} utility, to
both convert the trace into IR for and then to generate an STP formula
given the constraints on the symbolic input. The invocation looks
like:
\begin{Verbatim}[fontsize=\relsize{0}, frame=lines, framesep=.5em]
% ./utils/appreplay -trace /tmp/foo.trace -use-thunks false -use-post-var true \
  -stp-out foo.stp -ir-out foo.ir -wp-out foo.wp
...
Time to create sym constraint from TM: 0.480301
\end{Verbatim}

This command line produces the final stp file as \verb'foo.stp', and
the intermediate files \verb'foo.ir' and \verb'foo.wp' for aiding
explanation and understanding of the internals.  Remember that VinE
uses its own IR to model the exact semantics of instructions in a
simpler RISC-like form.  The IR and WP output files are in this IR
language. If you aren't interested in these files, you can omit the
\verb'-ir-out' and \verb'-wp-out' options.

Intuitively, \texttt{appreplay} models the logic of the executed
instructions, generating a path constraint needed to force the
execution down the path taken in the trace.  A variable \verb'post' is
introduced, which is the conjunction of the conditions seen along the
path. In the file \verb'foo.ir', you can see this variable is assigned
at each conditional branch points as $post = post \wedge condition$,
where a condition is a variable modeling the compare operation's
result that must be true to force execution to continue along the path
taken. (Because the language is explicitly typed and \verb'appreplay'
is careful to generate unique names, the full name of the \verb'post'
variable is likely something like \verb'post_1034:reg1_t', where the
part after the colon tells you it's a one-bit (boolean) variable.)

This weakest precondition formula is then converted to the format of
the STP solver's input.

This  Vine utility  has several  options which  are detailed  later in
Section~\ref {sec:utils}. The ones important here are:

\begin {itemize}
  \item \texttt{-trace} $\langle\text{\em file\/}\rangle$.
  Specifies the trace input file.

  \item \texttt{-use-thunks} $\langle\text{\em bool\/}\rangle$. For many
  analyses in BitBlaze, the implicit effects of the instruction are
  irrelevant.  It is sometimes not useful to generate complicated
  side-effect VINE IR inline in the IR statements of instructions.
  Instead, we can ask VinE to generate calls to these flag computation
  IR ``thunk'' functions that model the flag operations.  In our case,
  we do need the flags information to be present inline for the STP
  formula generation to interpret implicit dependences between compare
  and branch instructions. Hence, we have disabled the thunk
  generation.

  \item \texttt{-use-post-var} $\langle\text{\em bool\/}\rangle$.
  If this is set to true, then 'assert' statements will be rewritten
  to update a variable 'post', such that at the end of the trace
  'post' will have value 1 iff all assertions would have passed.

\end {itemize}


\section {Querying STP}

Now, in the last step we wish to ask the question ``what input values
force the execution down the path taken in the execution?''.  In the
formula we've built, this is equivalent to asking for a set of
assignments that make the variable \verb'post' true. We use STP to
solve this for us.  Notice that the STP file has the symbolic
\verb'INPUT' variable marked free in the final formula.

A symbolic formula $F$ is \emph{valid} iff it is true in all
interpretations.  In other words, $F$ is valid iff all assignments to
the free (symbolic) variables make $F$ true. Given a formula STP
decides whether it is valid or not. If it is invalid, then there
exists at least one set of inputs that make the formula false, then
STP gives such an assignment (a {\em counterexample}). We use this
trick to get the assignment to the free \verb'INPUT' variable in the
formula that makes the execution follow the traced path.

To do this, we add the following 2 lines at the end of the STP file
and run STP on it:

\begin{Verbatim}[fontsize=\relsize{0}, frame=lines, framesep=.5em]
% cat >>foo.stp
QUERY(FALSE);
COUNTEREXAMPLE;
% ./stp/stp foo.stp
Invalid.
ASSERT( INPUT_1001_0_41  = 0hex35  );
\end{Verbatim}

STP's reply of \verb'Invalid.' indicates it has determined that the
query formula \verb'FALSE' is not valid: there is an assignment to the
program inputs that satisfies the other assertions in the file (i.e.,
would lead the program to execute the same path that was observed),
but still leaves \verb'FALSE' false. As a counterexample it gives one
such input (in this case, the only possible one), in which the input
has the hex value \verb'0x35' (ASCII for \texttt{5}).



\section {Documentation of various utilities in VinE}
\label{sec:utils}

Here is a slightly more detailed explanation of the VinE utilities
used in this exercise.

\subsection {Appreplay}

\begin{itemize}

\item \texttt{-trace} : specifies the TEMU execution trace file to process

\item \texttt {-state} and \texttt{-state-range} are used to
initialize ranges of memory locations from a TEMU state snapshot.

\item \texttt{-conc-mem-idx} is an optimization to do some constant
propagation, which appears to help STP quite a bit. This will likely
become deprecated once some of the STP optimization issues are
resolved.

\item \texttt{-prop-consts} is another optimization that propagates
all constant values using VinE's evaluator.

\item \texttt {-use-thunks} if set to true, the generated IR will
have calls to functions to update the processor's condition codes
(\verb'EFLAGS' for the x86). If false, this code will be inlined
instead.  For most analysis purposes this should be disabled. It may
be useful for generating a smaller IR with the intent of giving it to
the evaluator rather than to STP.

\item \texttt{-use-post-var} if this is  set to true, then
\verb'assert' statements will be rewritten to update a variable
'post', such that at the end of the trace \verb'post' will have value
true if and only if all assertions would have passed.  This is mostly
for backwards compatibility for before we introduced the \verb'assert'
statement.

\item \texttt  {-deend} performs "deendianization",  i.e. rewrites
all memory expressions to equivalent array expressions. This should
usually be enabled.

\item \texttt {-concrete} initializes all the 'input' symbols to
the values they had in the trace.

\item \texttt{-verify-expected}  is  mostly  for regression/sanity  tests,  in
conjunction with \verb'-concrete'. \texttt{-verify-expected} adds
assertions to verify the all operands subsequently computed from those
symbols have the same value as they did in the trace, as they should
in this case.

\item \texttt{-include-all} translates and includes \emph{all} instructions,
rather than only those that (may) operate on tainted data. Generally
not desirable, but sometimes useful for debugging.

\item \texttt{-ir-out} specify the output ir file.

\item \texttt{-wp-out} and  \texttt{-stp-out} tell  appreplay to
compute the weakest precondition (WP) over the variable \verb'post'
(described above), and convert the resulting IR to an STP formula. the
formula holds for inputs that would follow the same execution path as
in the trace.

\end{itemize}



\section {Reporting Bugs}

Please report bugs to the bugzilla at:
\texttt{https://bullseye.cs.berkeley.edu/bugzilla/}.

VinE contains a \texttt{VERSION} file in its source base
directory. Please report the version number from that file when filing
bugs. And please also report if you notice something wrong or out of
date in this document.

\end{document}
