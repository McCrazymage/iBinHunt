\section {Documentation of various utilities in VinE}
\label{sec:utils}

Here is a slightly more detailed explanation of the VinE utilities
used in this exercise.

\subsection {Appreplay}

\begin{itemize}

\item \texttt{-trace} : specifies the TEMU execution trace file to process

\item \texttt {-state} and \texttt{-state-range} are used to
initialize ranges of memory locations from a TEMU state snapshot.

\item \texttt{-conc-mem-idx} is an optimization to do some constant
propagation, which appears to help STP quite a bit. This will likely
become deprecated once some of the STP optimization issues are
resolved.

\item \texttt{-prop-consts} is another optimization that propagates
all constant values using VinE's evaluator.

\item \texttt {-use-thunks} if set to true, the generated IR will
have calls to functions to update the processor's condition codes
(\verb'EFLAGS' for the x86). If false, this code will be inlined
instead.  For most analysis purposes this should be disabled. It may
be useful for generating a smaller IR with the intent of giving it to
the evaluator rather than to STP.

\item \texttt{-use-post-var} if this is  set to true, then
\verb'assert' statements will be rewritten to update a variable
'post', such that at the end of the trace \verb'post' will have value
true if and only if all assertions would have passed.  This is mostly
for backwards compatibility for before we introduced the \verb'assert'
statement.

\item \texttt  {-deend} performs "deendianization",  i.e. rewrites
all memory expressions to equivalent array expressions. This should
usually be enabled.

\item \texttt {-concrete} initializes all the 'input' symbols to
the values they had in the trace.

\item \texttt{-verify-expected}  is  mostly  for regression/sanity  tests,  in
conjunction with \verb'-concrete'. \texttt{-verify-expected} adds
assertions to verify the all operands subsequently computed from those
symbols have the same value as they did in the trace, as they should
in this case.

\item \texttt{-include-all} translates and includes \emph{all} instructions,
rather than only those that (may) operate on tainted data. Generally
not desirable, but sometimes useful for debugging.

\item \texttt{-ir-out} specify the output ir file.

\item \texttt{-wp-out} and  \texttt{-stp-out} tell  appreplay to
compute the weakest precondition (WP) over the variable \verb'post'
(described above), and convert the resulting IR to an STP formula. the
formula holds for inputs that would follow the same execution path as
in the trace.

\end{itemize}
